\documentclass[aspectratio=169]{beamer}
% \usecolortheme{beaver}
% \setbeamertemplate{navigation symbols}{}

% \usepackage[main=italian, english]{babel}
% \usepackage[utf8]{inputenc}
% \usepackage[T1]{fontenc}

\usepackage{comment}
\usepackage{pdfpages}

\usepackage{tikz}
\usepackage{pgfpages}
\pgfpagesuselayout{8 on 1}[a4paper, border shrink=5mm]
% \setbeamertemplate{background canvas}{
% 	\tikz \draw (current page.north west) rectangle (current page.south east);
% }

% arara: pdflatex: { shell: yes }
% arara: latexmk: { clean: partial }
% arara: clean: { extensions: [nav, snm] }
\begin{document}
\setbeamercolor{background canvas}{bg=}

% \includepdf[pages={4-7, 11-12}]{assets/rseba/00_PRESENTAZIONE_HANDOUTS.pdf}
% \includepdf[pages={2-3, 6, 8-16, 17-19}]{assets/rseba/01_CONCETTI-ELEMENTARI_HANDOUTS.pdf}
% \includepdf[pages={2-5, 6-11, 16-19, 27-39, 41-42, 44-45, 47-57}]{assets/rseba/02_VARIABILI-COSTANTI-TIPI_HANDOUTS.pdf}
% \includepdf[pages={2-9, 11-12, 14, 17, 19-20, 27-31}]{assets/rseba/03_ISTRUZIONI_HANDOUTS.pdf}

\begin{comment}
\inlcudepdf[pages={}]{rseba/}
\includepdf[pages={2-9, 11-19, 21-23, 25, 28, 30-33, 35, 37-39, 41}]{rseba/05_FUNZIONI_HANDOUTS.pdf}
\end{comment}

\end{document}
