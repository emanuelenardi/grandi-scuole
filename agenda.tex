\documentclass[twoside, symmetric]{tufte-book}
% \documentclass[gray]{tufte-book}
% NOTE these two lines resolves the "too many unprocessed floats" error.
\geometry{bottom=3cm}
\usepackage{morefloats}

\usepackage{comment}
\usepackage{xparse}

\usepackage[main=italian, english]{babel}
\usepackage[
	calc,
	useregional,
	showdow,
	showseconds = false,
	% ,timesep     = {.}
]{datetime2}
\usepackage{booktabs}
\usepackage{multirow}
\usepackage{array}
\usepackage{mparhack, marginfix}

\usepackage{afterpage}
\newcommand{\blankpage}{%
	\clearpage%
	\begingroup%
		\thispagestyle{empty}%
		\addtocounter{page}{-1}%
		\vspace*{\stretch{4}}%
		\centering%
		% Questa pagina è stata lasciata vuota intenzionalmente%
		\vspace*{\stretch{1}}%
		\clearpage%
    \endgroup%
}

\usepackage[inline]{enumitem}
\usepackage{amssymb}
\NewDocumentCommand{\cmark}{}{\ding{51}}
\NewDocumentCommand{\xmark}{}{\ding{55}}

% \newlist{todolist}{itemize}{2}
% \setlist[todolist]{label=\(\square\)}
% \NewDocumentCommand{\done}{}{\rlap{\(\square\)}{\raisebox{2pt}{\large\hspace{1pt}\cmark}}%
% \hspace{-2.5pt}}
% \NewDocumentCommand{\notdone}{}{\rlap{\(\square\)}{\large\hspace{1pt}\xmark}}

\usepackage{menukeys}
% NOTE ridefinizione del simbolo per indicare i sottomenù
\newmenustylesimple*{arrows}{\CurrentMenuElement}[ \(\rightarrow\) ]{blacknwhite}
\AtBeginDocument{\renewmenumacro{\menu}[>]{arrows}}

\usepackage{pifont}
\NewDocumentCommand{\Presente}{}{\textcolor{ForestGreen}{\cmark}}
\NewDocumentCommand{\Assente}{}{\textcolor{Red}{\xmark}}
\NewDocumentCommand{\Festa}{}{\textcolor{gray}{\(\circ\)}}
\NewDocumentCommand{\Null}{}{-}

% https://tex.stackexchange.com/questions/477319/
\DTMnewdatestyle{giorno-numero-mese}{%
	\renewcommand{\DTMdisplaydate}[4]{%
		\DTMweekdayname{##4}\space%  weekday
		\number##3\nobreakspace%     day
		\DTMmonthname{##2}%         (full) Month
	}%
	\renewcommand{\DTMDisplaydate}{\DTMdisplaydate}%
}

\DTMnewdatestyle{giorno-numero}{%
	\renewcommand{\DTMdisplaydate}[4]{%
		\DTMweekdayname{##4}\space% weekday
		\number##3%                 day
	}%
	\renewcommand{\DTMDisplaydate}{\DTMdisplaydate}%
}

\DTMnewdatestyle{numero}{%
	\renewcommand{\DTMdisplaydate}[4]{%
		\number##3%                 day
	}%
	\renewcommand{\DTMDisplaydate}{\DTMdisplaydate}%
}

\NewDocumentCommand{\Data}{m}{%
	\DTMsetdatestyle{giorno-numero-mese}%
	\DTMdate{#1}%
}

\NewDocumentCommand{\Giorno}{m}{%
	\DTMsetdatestyle{giorno-numero}%
	\DTMdate{#1}%
}

\NewDocumentCommand{\Numero}{m}{%
	\DTMsetdatestyle{numero}%
	\DTMdate{#1}%
}

\NewDocumentEnvironment{lezione}{mmmm}{
	\subsection{\textbf{#4}}
	\marginnote{\Data{#1}}%\Giorno{#1}
	\marginnote{\DTMtime{#2} -- \DTMtime{#3}}%
}{
	% \vspace{1em}
}

% \NewDocumentCommand{\studente}{m}{%
% 	\paragraph{#1}%
% }

\NewDocumentEnvironment{compiti}{}{%
	\begin{todolist}\footnotesize
}{
	\end{todolist}
}

\NewDocumentCommand{\Cpp}{}{C\texttt{++}}

\title{Diario delle lezioni}
\author{Emanuele Nardi}
\date{Aggiornato al \today}

% arara: pdflatex: { draft: yes }
% arara: pdflatex: { synctex: yes }
% arara: latexmk: { clean: partial }
% arara: clean: { extensions: [bbl, synctex.gz] }
\begin{document}

\maketitle
\blankpage

\section{Introduzione}

Per ogni lezione sono espresse in maniera sintetica i punti salienti di ogni lezione.%, per leggere una traccia gli argomenti trattati consultare la dispensa.

\medskip
Il diario dell'insegnante è un diario dove l'insegnante registra lo svolgimento della lezione ed esprime le sue riflessioni su di essa. \'{E} utilizzato come \emph{strumento di sviluppo}. Leggendola il genitore può farsi un'idea più precisa del coinvolgimento del proprio figlio all'interno del percorso di apprendimento.

\medskip
\begin{flushright}
Il tutor\\
\emph{Emanuele Nardi}
\end{flushright}
\afterpage{\blankpage}
\clearpage

\section*{Tabelle riassuntive presenze}

Per ogni giorno di lezione è segnata la presenza (\Presente{}), l'assenza (\Assente{}) e giorno di festività/ponte (\Festa{}).
Alla destra del grafico sono indicate la somma delle ore in cui lo studente era presente o assente rispettivamente.
Infine è indicata una percentuale delle ore di presenza.
Nel caso siano presenti più spunti sta a significare che in quella data si sono svolte più lezioni (mattina e pomeriggio ad esempio).

\begin{table}[h]
\raggedright
\begin{tabular}{@{} l *{10}{c} @{}}
	\toprule
		\multirow{2}{*}{Aprile} & \multicolumn{9}{c}{Presenze} \\
	\cmidrule(l){2-11}
		& \Numero{2019-04-08}
		& \Numero{2019-04-12}
		& \Numero{2019-04-15}
		& \Numero{2019-04-19}
		& \Numero{2019-04-22}
		& \Numero{2019-04-26}
		& \Numero{2019-04-29}
		& \Presente & \Assente & \% \\
	\cmidrule(r){0-7}
	\cmidrule(l){9-11}
		Condini M. & \Presente & \Presente & \Assente & \Festa & \Festa & \Festa & \Assente & 6 & 6 & 50 \\
		Ferreira A. & \Assente & \Presente & \Assente & \Festa & \Festa & \Festa & \Assente & 9 & 3 & 25 \\
		Malfatti N. & \Presente & \Presente & \Presente & \Festa & \Festa & \Festa & \Presente & 12 & 12 & 100 \\
		Rama F. & \Presente & \Presente & \Assente & \Festa & \Festa & \Festa & \Presente & 9 & 3 & 75 \\
	\bottomrule
\end{tabular}
\end{table}
\bigskip

\begin{table}[h]
\raggedright
\begin{tabular}{@{} l *{15}{c} @{}}
	\toprule
		\multirow{2}{*}{Maggio} & \multicolumn{14}{c}{Presenze} \\
	\cmidrule(l){2-16}
		& \Numero{2019-05-03}
		& \Numero{2019-05-06}
		& \Numero{2019-05-10}
		& \Numero{2019-05-13}
		& \Numero{2019-05-17}
		& \Numero{2019-05-18}
		& \Numero{2019-05-20}
		& \Numero{2019-05-24}
		& \Numero{2019-05-25}
		& \Numero{2019-05-27}
		& \Numero{2019-05-28}
		& \Numero{2019-05-31}
		& \Presente & \Assente & \% \\
	\cmidrule(r){0-12}
	\cmidrule(l){14-16}
		Condini M. & \Assente & \Assente & \Assente & \Assente & \Assente & \Assente & \Assente  & \Assente & \Assente & \Assente & \Assente & \Assente\Assente & 0 & 39 & 0 \\
		Ferreira A. & \Presente & \Assente & \Assente & \Assente & \Assente & \Assente & \Presente & \Assente & \Assente & \Assente & \Assente & \Assente\Assente & 6 & 33 & 15 \\
		Malfatti N. & \Assente & \Assente & \Presente & \Presente & \Presente & \Presente & \Presente & \Presente & \Presente & \Presente & \Presente & \Presente\Presente & 33 & 6 & 85 \\
		Rama F. & \Presente & \Presente & \Presente & \Assente & \Presente & \Presente & \Presente  & \Presente & \Presente & \Presente & \Presente & \Presente\Presente & 36 & 3 & 92 \\
	\bottomrule
\end{tabular}
\end{table}
\bigskip

\begin{table}[h]
\raggedright
\begin{tabular}{@{} l *{6}{c} @{}}
	\toprule
		\multirow{2}{*}{Giugno} & \multicolumn{4}{c}{Presenze} \\
	\cmidrule(l){2-7}
		& \Numero{2019-06-03}
		& \Numero{2019-06-04}
		& \Numero{2019-06-05}
		& \Presente & \Assente & \% \\
	\cmidrule(r){0-3}
	\cmidrule(l){5-7}
		Condini M. & \Null & \Null & \Null & 0 & 0 & 0 \\
		Ferreira A. & \Null & \Null & \Null & 0 & 0 & 0 \\
		Malfatti N. & \Null & \Null & \Null & 0 & 0 & 0 \\
		Rama F. & \Null & \Null & \Null & 0 & 0 & 0 \\
	\bottomrule
\end{tabular}
\end{table}
\bigskip

\section*{Aprile 2019}

\begin{lezione}
	{2019-04-08}
	{09:00:00}{12:00:00}
	{Fondamenti di Unix}

\begin{itemize}
	\item impostato un luogo di sviluppo comune su Windows. Installando Atom come editor di testo e la WSL (\emph{Windows Subsystem for Linux}) per la compilazione come su macchine Unix;
	\item imparato a
	\begin{enumerate*}
		\item sapere qual è l'utente loggato (\texttt{whoami});
		\item stampare la cartella corrente (\texttt{pwd});
		\item navigare fra le cartelle (\texttt{cd});
		\item creare cartelle (\texttt{mkdir});
		\item creare file (\texttt{touch});
		\item spostare e rinominare file (\texttt{mv});
		\item rendere eseguibile un file (\texttt{chmod +x})
	\end{enumerate*} tramite i comandi da terminale
	\item imparato a scrivere uno script per eseguire codice a scelta (file \texttt{compile.sh});
	\item ripassato la procedura per compilare i file in linguaggio {\Cpp} (tramite \texttt{g++} e WSL);
\end{itemize}

\paragraph{Diario dell'insegnante}
Gli studenti sono stati molto reattivi durante gli esercizi di pratica e attenti durante la spiegazione teorica.
Mentre hanno dimostrato facilmente di distrarsi durante i tempi morti.

% \studente{Michele}
% Risulta comprendere con relativa facilità le nozioni; pone domande per chiedere precisazioni sull'argomento.
%
% \studente{Nicholas}
% Presenta delle difficoltà nell'esprimere concetti difficili in modo conciso.
%
% \studente{Francesko}
% Non è intervenuto molto in questa lezione.

\end{lezione}

\begin{lezione}
	{2019-04-12}
	{09:00:00}{12:00:00}
	{Fondamenti di \Cpp}

\begin{itemize}
	\item Lezione ``concetti elementari'';
	\item Lezione ``variabili, costanti, tipi'';
	\item Lezione ``variabili e costanti''.
\end{itemize}

\paragraph{Diario dell'insegnante}
La classe era evidentemente annoiata dal comportamento di Alan per l'intera durata della lezione frontale.

% \studente{Alan}
% Ha avuto un comportamente scontroso ed interveniva senza essere interpellato.

\end{lezione}

\begin{lezione}
	{2019-04-15}
	{09:00:00}{12:00:00}
	{Istruzioni di \Cpp}

\begin{itemize}
	\item Lezione ``istruzioni'';
	\begin{itemize}
		\item \texttt{if-then}, \texttt{if-then-else}, \texttt{switch}, \texttt{while}, \texttt{do-while}, \texttt{for};
		\item \texttt{continue}, \texttt{break}, \texttt{goto}
	\end{itemize}
\end{itemize}

\paragraph{Diario dell'insegnante}
Data l'assenza della maggior parte della classe la lezione è finita prima del previsto, lasciando così il tempo alla visione degli esercizi.

% \studente{Nicholas}
% \'E stato attento durante la lezione ed ha eseguito gli esercizi quando richiesto.
% Ha mostrato interesse per la materia.

\end{lezione}

\begin{lezione}
	{2019-04-29}
	{09:00:00}{12:00:00}
	{Puntatori e Riferimenti}

\begin{itemize}
	\item Lezione ``puntatori riferimenti'';
	\begin{itemize}
		\item Il tipo ``riferimento a'', vincoli sull'uso dei riferimenti;
		\item L'operatore address-of ``\texttt{\&}'';
		\item Il tipo ``puntatore a'';
		\item L'operatore di dereference ``\texttt{*}'';
		\item Assegnazione fra puntatori.
	\end{itemize}
\end{itemize}

\paragraph{Diario dell'insegnante}
Nella prima ora abbiamo ripassato gli argomenti trattati nell'ultima lezione, data l'assenza della maggior parte degli studenti alla lezione precedente.
Nelle due ore successive abbiamo svolto un'alternanza fra lezione frontale ed esercitazione, svolgendo gli esercizi proposti in modo guidato.

% \studente{Nicholas}
% Inizia a prendere confidenza con la materia e dimostra sicurezza nel dare le risposte.
%
% \studente{Francesko}
% Non era in grado di svolgere gli esercizi, ha bisogno di applicarsi.

\end{lezione}

\clearpage
\section*{Maggio 2019}

\begin{lezione}
	{2019-05-03}
	{09:00:00}{12:00:00}
	{Ripasso costrutti in C}

Abbiamo effettuato un ripasso delle due lezioni precendenti.

\end{lezione}

\begin{lezione}
	{2019-05-06}
	{09:30:00}{12:30:00}
	{Funzioni in C -- parte 1}

	\begin{itemize}
		\item Lezione ``Funzioni'':
		\begin{itemize}
			\item Introduzione
			\item Definizione, Dichiarazione e Chiamata di Funzioni
		\end{itemize}
	\end{itemize}

\end{lezione}

\begin{lezione}
	{2019-05-10}
	{09:00:00}{12:00:00}
	{Funzioni in C -- parte 2}

	\begin{itemize}
		\item Lezione  ``Funzioni''
		\begin{itemize}
			\item Parametri e variabili locali
			\item Passaggio di parametri
			\item Ricorsione
		\end{itemize}
	\end{itemize}

\end{lezione}

\begin{lezione}
	{2019-05-13}
	{09:00:00}{12:00:00}
	{Vettori in C}

	\begin{itemize}
		\item Lezione ``Array'':
		\begin{itemize}
			\item Definizione ed Utilizzo di Array
			\item Array e Funzioni
			\item Array ordinati
			\item Array Multidimensionali
			\item Array e puntatori
		\end{itemize}
		% \item Array, Puntatori e Funzioni
	\end{itemize}

\end{lezione}

\begin{lezione}
	{2019-05-17}
	{09:00:00}{12:00:00}
	{Stringhe e I/O su file}

	\begin{itemize}
		\item Lezione ``Stringhe e File di testo'':
		\begin{itemize}
			\item Stringhe
			\item Argomenti da linea di comando
		\end{itemize}
	\end{itemize}

\end{lezione}

\begin{lezione}
	{2019-05-18}
	{09:00:00}{12:00:00}
	{Le Strutture}

	\begin{itemize}
		\item Lezione ``Stringhe e File di testo'':
		\begin{itemize}
			\item I/O con file di testo
		\end{itemize}

		\item Lezione ``Le Strutture'':
		\begin{itemize}
			\item Le Strutture
			\item Le operazioni sulle strutture
		\end{itemize}
	\end{itemize}

\end{lezione}

\begin{lezione}
	{2019-05-20}
	{09:00:00}{12:00:00}
	{Modularità del codice e Strutture Dati}

	\begin{itemize}
		\item Lezione ``Le Strutture'':
		\begin{itemize}
			\item Le Strutture ricorsive
			\item Array ordinati di strutture
		\end{itemize}
		\item Lezione ``La modularità'':
		\begin{itemize}
			\item Modello di gestione della memoria di un programma
			\item Programmazione su file multipli
		\end{itemize}
		\item Lezione ``Le Strutture Dati'':
		\begin{itemize}
			\item Tipo di dato astratto (Array, Liste concatenate, Grafi
			\item Realizzazione di Pila, Coda, Alberi Binari
		\end{itemize}
	\end{itemize}

\end{lezione}

\begin{lezione}
 	{2019-05-24}
 	{09:00:00}{12:00:00}
 	{Introduzione a Java}

	\begin{itemize}
		\item Differenze e somiglianze con la programmazione in C:
		\begin{enumerate*}[label={\arabic*.}]
			\item stampa (\texttt{sout})
			\item dichiarazione di variabili
			\item caratteri di escape (\texttt{\textbackslash{}n}, \texttt{\textbackslash{}t})
			\item dichiarazione degli array
			\item gestione della memoria dinamica, casting.
		\end{enumerate*}
		\item Concetto di oggetto, classe e istanza;
		\item Indicazioni di risorse per lo studio individuale.
	\end{itemize}

\end{lezione}

\begin{lezione}
	{2019-05-25}
	{09:00:00}{12:00:00}
	{Familiarizzazione con l'ambiente di lavoro}

Esplorazione delle funzionalità di NetBeans:
\begin{enumerate*}[label={\arabic*.}]
	\item cerca nel file (\texttt{ctrl\,+\,F})
	\item genera codice (\texttt{alt\,+\,insert})
	\item correggi tutte le importazioni di classe (\texttt{ctrl\,+\,shift\,+\,I})
	\item esegui progetto (\texttt{F6}, build + compile)
	\item formattazione del codice (\texttt{ctrl\,+\,shift\,+\,F})
	\item copia linea sopra/sotto (\texttt{ctrl\,+\,shift\,+ \arrowkeyup{} / \arrowkeydown{}} )
\end{enumerate*}

Scrittura veloce del codice tramite utilizzo di scorciatoie:
\begin{enumerate*}[label={\arabic*.}]
	\item \texttt{St} (String)
	\item \texttt{bo} (boolean)
	% \item \texttt{cl} (class)
	\item \texttt{do} (do-while)
	\item \texttt{for} (for)
	\item \texttt{fore} (foreach)
	\item \texttt{if}, \texttt{ifelse}
	% \item \texttt{im} (implements)
	% \item \texttt{pe} (protected)
	\item \texttt{pr} (private)
	\item \texttt{pu} (public)
	\item \texttt{re} (return)
	\item \texttt{sout} (print)
	\item \texttt{sw} (switch)
	\item \texttt{wh} (while).
\end{enumerate*}

Scrivere e produrre la documentazione del codice tramite \texttt{javadoc} \menu{Run > Generate Javadoc}.

\end{lezione}

\begin{lezione}
	{2019-05-27}
	{13:00:00}{16:00:00}
	{Stringhe e Vettori}

\texttt{String}, \texttt{StringBuilder}, Vettori.
Laboratorio.

\end{lezione}

\begin{lezione}
	{2019-05-28}
	{09:00:00}{12:00:00}
	{Funzioni e input}

Definizione di funzioni e input da terminale e da interfaccia grafica.

\end{lezione}

\begin{lezione}
	{2019-05-31}
	{09:00:00}{12:00:00}
	{Programmazione ad oggetti}

Scrittura di un diagramma UML.
Scrittura di classi: definizione dei costruttori.
Creazione degli oggetti: istanziare un oggetto, richiamarne i suoi metodi.
% Utilizzo delle interfacce.

\end{lezione}

\begin{lezione}
	{2019-05-31}
	{13:00:00}{16:00:00}
	{Ereditarietà, interfacce e poliformismo}

Da definire.

\end{lezione}

\clearpage
\section*{Giugno 2019}

\end{document}
