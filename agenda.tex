\documentclass{tufte-book}
\usepackage{comment}
\usepackage{xparse}

\usepackage[main=italian, english]{babel}
\usepackage[
	calc,
	useregional,
	showdow,
	showseconds = false,
	% ,timesep     = {.}
]{datetime2}
\usepackage{booktabs}
\usepackage{multirow}
\usepackage{array}
\usepackage{
	,mparhack
	,marginfix
}

\usepackage[inline]{enumitem}
\usepackage{amssymb}
\newlist{todolist}{itemize}{2}
\setlist[todolist]{label=\(\square\)}
\NewDocumentCommand{\cmark}{}{\ding{51}}%
\NewDocumentCommand{\xmark}{}{\ding{55}}%
\NewDocumentCommand{\done}{}{\rlap{\(\square\)}{\raisebox{2pt}{\large\hspace{1pt}\cmark}}%
\hspace{-2.5pt}}
\NewDocumentCommand{\notdone}{}{\rlap{\(\square\)}{\large\hspace{1pt}\xmark}}

\usepackage{pifont}
\NewDocumentCommand{\Presente}{}{\textcolor{ForestGreen}{\cmark}}
\NewDocumentCommand{\Assente}{}{\textcolor{Red}{\xmark}}
\NewDocumentCommand{\Festa}{}{\textcolor{gray}{\(\circ\)}}
\NewDocumentCommand{\Null}{}{-}

% https://tex.stackexchange.com/questions/477319/
\DTMnewdatestyle{giorno-numero}{%
	\renewcommand{\DTMdisplaydate}[4]{%
		\DTMweekdayname{##4}\space% weekday
		\number##3%                 day
	}%
	\renewcommand{\DTMDisplaydate}{\DTMdisplaydate}%
}

\DTMnewdatestyle{numero}{%
	\renewcommand{\DTMdisplaydate}[4]{%
		\number##3%                 day
	}%
	\renewcommand{\DTMDisplaydate}{\DTMdisplaydate}%
}

\NewDocumentCommand{\Giorno}{m}{%
	\DTMsetdatestyle{giorno-numero}%
	\DTMdate{#1}%
}

\NewDocumentCommand{\Numero}{m}{%
	\DTMsetdatestyle{numero}%
	\DTMdate{#1}%
}

\NewDocumentEnvironment{lezione}{mmmm}{
	\subsection{\textbf{#4}}%
	\marginnote{\Giorno{#1}}%
	\marginnote{\DTMtime{#2} -- \DTMtime{#3}}%
}{
	\vspace{0.5cm}
}

\NewDocumentCommand{\studente}{m}{%
	\paragraph{#1}%
}

\NewDocumentEnvironment{compiti}{}{%
	\begin{todolist}\footnotesize
}{
	\end{todolist}
}

\NewDocumentCommand{\Cpp}{}{C\texttt{++}}

\AtBeginDocument{\maketitle}
% \AtEndDocument{Continua\dots}

\title{Diario delle lezioni}
\author{Emanuele Nardi}
\date{Aggiornato al \today}

% arara: pdflatex: { draft: yes }
% arara: pdflatex: { synctex: yes }
% arara: latexmk: { clean: partial }
% arara: clean: { extensions: [bbl, synctex.gz] }
\begin{document}

\section{Introduzione}

All'inizio di ogni mese è presente una tabella riassuntiva delle presenze dello studente.

\medskip
Per ogni lezione sono espresse in maniera sintetica i punti salienti di ogni lezione, per approfondire gli argomenti trattati leggere la dispensa delle lezioni.

\medskip
Il diario dell'insegnante è un diario dove l'insegnante registra lo svolgimento della lezione ed esprime le sue riflessioni su di essa. \'{E} utilizzato come \emph{strumento di sviluppo}. Leggendola il genitore può farsi un'idea più precisa del conivolgimento del proprio figlio all'interno del percorso di apprendimento.

\medskip
\begin{flushright}
Il tutor\\
\emph{Emanuele Nardi}
\end{flushright}

\clearpage
\section*{Aprile 2019}

\begin{table}[h]
\centering
\begin{tabular}{@{} l *{7}{c} @{}}
	\toprule
		\multirow{2}{*}{Aprile} & \multicolumn{7}{c}{Presenze} \\
	\cmidrule(l){2-8}
		& \Numero{2019-04-08} & \Numero{2019-04-12} & \Numero{2019-04-15} & \Numero{2019-04-19} & \Numero{2019-04-22} & \Numero{2019-04-26} & \Numero{2019-04-29} \\
	\midrule
		Alan & \Assente & \Presente & \Assente & \Festa & \Festa & \Festa & \Assente \\
		Francesko & \Presente & \Presente & \Assente & \Festa & \Festa & \Festa & \Presente \\
		Michele & \Presente & \Presente & \Assente & \Festa & \Festa & \Festa & \Assente \\
		Nicholas & \Presente & \Presente & \Presente & \Festa & \Festa & \Festa & \Presente \\
	\bottomrule
\end{tabular}
\end{table}
\bigskip

\subsection*{Settimana 8 -- 13 aprile}

\begin{lezione}
	{2019-04-08}
	{09:00:00}{12:00:00}
	{Fondamenti di Unix}

\begin{itemize}
	\item impostato un luogo di sviluppo comune su Windows. Installando Atom come editor di testo e la WSL (\emph{Windows Subsystem for Linux}) per la compilazione come su macchine Unix;
	\item imparato a
	\begin{enumerate*}
		\item sapere qual è l'utente loggato (\texttt{whoami});
		\item stampare la cartella corrente (\texttt{pwd});
		\item navigare fra le cartelle (\texttt{cd});
		\item creare cartelle (\texttt{mkdir});
		\item creare file (\texttt{touch});
		\item spostare e rinominare file (\texttt{mv});
		\item rendere eseguire un file (\texttt{chmod +x})
	\end{enumerate*} tramite i comandi da terminale
	\item imparato a scrivere uno script per eseguire codice a scelta (file \texttt{compile.sh});
	\item ripassato la procedura per compilare i file in linguaggio {\Cpp} (tramite \texttt{g++} e WSL);
\end{itemize}

\paragraph{Diario dell'insegnante}
Gli studenti sono stati molto reattivi durante gli esercizi di pratica e attenti durante la spiegazione teorica.
Mentre hanno dimostrato facilmente di distrarsi durante i tempi morti.

\studente{Michele}
Risulta comprendere con relativa facilità le nozioni; pone domande per chiedere precisazioni sull'argomento.

\studente{Nicholas}
Presenta delle difficoltà nell'esprimere concetti difficili in modo conciso.

\studente{Francesko}
Non è intervenuto molto in questa lezione.
La prossima lezione non si deve sedere vicino a Nicholas (si distraggono a vicenda).

\end{lezione}

\begin{lezione}
	{2019-04-12}
	{09:00:00}{12:00:00}
	{Fondamenti di \Cpp}

\begin{itemize}
	\item Lezione ``concetti elementari'';
	\item Lezione ``variabili, costanti, tipi'';
	\item Lezione ``variabili e costanti''.
\end{itemize}

\paragraph{Diario dell'insegnante}

La classe era evidentemente annoiata dal comportamento di Alan per l'intera durata della lezione frontale.

\studente{Alan}
Ha avuto un comportamente scontroso ed interveniva senza essere interpellato.

\end{lezione}

\begin{lezione}
	{2019-04-15}
	{09:00:00}{12:00:00}
	{Istruzioni di \Cpp}

\begin{itemize}
	\item Lezione ``istruzioni'';
	\begin{itemize}
		\item \texttt{if-then}, \texttt{if-then-else}, \texttt{switch}, \texttt{while}, \texttt{do-while}, \texttt{for};
		\item \texttt{continue}, \texttt{break}, \texttt{goto}
	\end{itemize}
\end{itemize}

\paragraph{Diario dell'insegnante}
Data l'assenza della maggior parte della classe la lezione è finita prima del previsto, lasciando così il tempo alla visione degli esercizi.

\studente{Nicholas}
\'E stato attento durante la lezione ed ha eseguito gli esercizi quando richiesto.
Ha mostrato interesse per la materia.

\end{lezione}

\begin{lezione}
	{2019-04-29}
	{09:00:00}{12:00:00}
	{Puntatori e Riferimenti}

\begin{itemize}
	\item Lezione ``puntatori riferimenti'';
	\begin{itemize}
		\item Il tipo ``riferimento a'', vincoli sull'uso dei riferimenti;
		\item L'operatore address-of ``\texttt{\&}'';
		\item Il tipo ``puntatore a'';
		\item L'operatore di dereference ``\texttt{*}'';
		\item Assegnazione fra puntatori.
	\end{itemize}
\end{itemize}

\paragraph{Diario dell'insegnante}
Nella prima ora abbiamo ripassato il programma svolto nell'ultima lezione, data l'assenza della maggior parte degli studenti.
Nelle due ore successive abbiamo svolto un'alternanza fra lezione frontale ed esercitazione, svolgendo gli esercizi in modo guidato.

\studente{Nicholas}
Inizia a prendere confidenza con la materia e dimostra sicurezza nel dare le risposte.

\studente{Francesko}
Non era in grado di svolgere gli esercizi, ha bisogno di applicarsi.

\end{lezione}


\clearpage
\section*{Maggio 2019}

\begin{table}[h]
\centering
\begin{tabular}{@{} l *{9}{c} @{}}
	\toprule
		\multirow{2}{*}{Maggio} & \multicolumn{9}{c}{Presenze} \\
	\cmidrule(l){2-10}
		& \Numero{2019-05-03} & \Numero{2019-05-6} & \Numero{2019-05-10} & \Numero{2019-05-13} & \Numero{2019-05-17} & \Numero{2019-05-20} & \Numero{2019-05-24} & \Numero{2019-05-27} & \Numero{2019-05-31} \\
	\midrule
		Alan & \Assente & \Null & \Null & \Null & \Null & \Null & \Null  & \Null & \Null \\
		Francesko & \Presente & \Null & \Null & \Null & \Null & \Null & \Null  & \Null & \Null \\
		Michele & \Assente & \Null & \Null & \Null & \Null & \Null & \Null  & \Null & \Null \\
		Nicholas & \Assente & \Null & \Null & \Null & \Null & \Null & \Null  & \Null & \Null \\
	\bottomrule
\end{tabular}
\end{table}
\bigskip

\end{document}
