\documentclass[12pt]{beamer}
\setbeamercovered{transparent}
\setbeamercolor{alerted text}{fg=orange}
\setbeamertemplate{navigation symbols}{}

\usepackage{slides.preamble}

% \usepackage{tikz}
% \usepackage{pgfpages}
% \pgfpagesuselayout{8 on 1}[a4paper, border shrink=5mm]
% \setbeamertemplate{background canvas}{
% 	\tikz \draw (current page.north west) rectangle (current page.south east);
% }

% NOTE informazioni incluse nella pagina del titolo
\title{Pillole di {\javalogo}}
\subtitle{(con un po' di zucchero)}
% \title{Pillole di {\javalogo}\\[5pt]\small(con un po' di zucchero)}
\author{Emanuele Nardi}
\institute{Grandi Scuole}
\date{\today}

\includeonlyframes{%
	title,%
	hello-world,%
	definition-italian%
}

% arara: xelatex: { shell: yes }
%! arara: xelatex: { shell: yes }
% NOTE non cancellare fls - è un file audio
% arara: clean: { extensions: [nav, snm, xdv, vrb] }
% arara: latexmk: { clean: partial }
\begin{document}

\begin{frame}[label=title]
\maketitle
\end{frame}

\begin{frame}[fragile, label=hello-world]
\javafile{assets/codes/java/HelloWorld.java}

\begin{uncoverenv}<+(1)->
\vfill
\begin{minted}[frame=none, fontsize=\footnotesize]{bash}
javac FirstJavaProgram.java
\end{minted}
\end{uncoverenv}

\uncover<+(1)-3>{
\begin{itemize}
	\item \href{https://www.fabriziorocca.it/guide/come-risolvere-il-problema-di-javac-non-riconosciuto-come-comando/}{``javac'' non è riconosciuto come comando}
\end{itemize}
}
\begin{uncoverenv}<+-3>
\begin{minted}[frame=none, fontsize=\footnotesize]{bash}
set path=C:/Program Files/Java/jdk1.8.0_<versione>/bin
\end{minted}
\end{uncoverenv}

\begin{uncoverenv}<+->
\begin{minted}[frame=none, fontsize=\footnotesize]{bash}
java FirstJavaProgram
\end{minted}
\end{uncoverenv}

\end{frame}

\begin{frame}[label=definition-italian, fragile]
\frametitle{Definizione di Java}
\only<0->{
In informatica {\java} è un linguaggio di programmazione \alert<2>{\textbf{ad alto livello}}, \alert<3>{\textbf{orientato agli oggetti}} e a \alert<4->{\textbf{tipizzazione statica}}.\\[10pt]%
}

\only<+>{%
Si appoggia sull'omonima piattaforma software, specificamente progettato per essere il più possibile \textbf{indipendente dalla piattaforma hardware di esecuzione} (tramite compilazione in bytecode prima e interpretazione poi da parte di una JVM).
\source{Wikipedia (in italiano)}{https://it.wikipedia.org/wiki/Java_(linguaggio_di_programmazione)}
}

\only<+>{%
Un linguaggio di programmazione ad alto livello è un linguaggio di programmazione caratterizzato da una \alert{significativa astrazione} dai dettagli del funzionamento di un calcolatore e dalle caratteristiche del linguaggio macchina.%
\vfill
}%

\only<+>{%
In informatica la programmazione orientata agli oggetti (OOP, Object Oriented Programming) è un \alert{paradigma di programmazione} che \emph{permette di definire oggetti software in grado di interagire gli uni con gli altri} attraverso lo scambio di messaggi.%
\vfill
}%

\only<+>{%
La tipizzazione statica è una \alert{particolare politica di tipizzazione}, ovvero di assegnazione di tipi alle variabili.\\[5pt]%
Nei linguaggi a tipizzazione statica, \alert{il tipo di ogni variabile viene stabilito direttamente nel codice sorgente} dove viene assegnato esplicitamente per mezzo di parole chiave apposite, come ad esempio \texttt{int}, \texttt{long}, \texttt{float}, \texttt{char}
\vfill
}%

\only<+->{%
\javafile{assets/codes/java/EsempioTipizzazione.java}
\vfill
\only<+>{
Avviene un errore \textbf{in fase di compilazione}
}
}

\end{frame}

\end{document}
